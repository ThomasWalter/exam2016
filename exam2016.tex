\documentclass[11pt,addpoints]{exam}
\printanswers

\usepackage{amsmath}
\usepackage{amssymb}
\usepackage{bm}

\usepackage{graphicx}
\usepackage[usenames,dvipsnames]{color}

\usepackage[utf8]{inputenc}
\usepackage[T1]{fontenc}
\usepackage{lmodern} % load a font with all the characters


% Bold the 'Figure #' in the caption and separate it with a period
% Captions will be left justified
\usepackage[labelfont=bf,labelsep=period,justification=raggedright]{caption}
\usepackage{here}

\title{Examen Cours ES2: Introduction à la bioinformatique et la génomique}
\author{Chloé-Agathe Azencott, Thomas Walter}
\begin{document}
\maketitle 

\begin{questions}

\section{Questions sur le cours}

\question[4] Notions en Biologie. 
\begin{parts}
\part Citer trois fonctions de protéines. 
\part Expliquer pourquoi un codon est constitué de trois nucléotides. 
\part Expliquer le principe de l'interférence par ARN.
\part Classer les mutations selon leur effet. 
\end{parts}

---- THE REST HAS NOT BEEN TOUCHED. 

\question[2] La microscopie.
\begin{parts}
\part Décrire une technique pour mettre en évidence des structures
dans la cellule. 
\part Quelle est le principe de la microscopie par feuille de lumière
et quel est son principal avantage~? 
\part Quel est le principe et l'objectif de la microscopie par
illumination structurée~?  
\end{parts}

\question[2] Vous êtes consulté comme un expert en analyse d'image par
un biologiste qui veut reconnaître les morphologies montrées dans la
figure \ref{fig:morphologies}. Proposez des descripteurs qui seront
capables de distinguer les
morphologies d'en haut de celles d'en bas pour les colonnes (a), (b),
(c). 
\begin{figure}[h!]
\centering
%\includegraphics[scale=0.5]{morphologies_raw.png}
\caption{Classes morphologiques à distinguer}
\label{fig:morphologies}
\end{figure}

\question[2] Clustering. La figure \ref{fig:micro_array} montre le
résultat d'un clustering d'expression des gènes de plusieurs échantillons de
cancer du sein. Chaque ligne correspond à un gène, chaque colonne
correspond à un échantillon.  
\begin{figure}[h!]
\centering
%\includegraphics[scale=0.8]{microarray.pdf}
\caption{Microarray (extrait) : chaque ligne correspond à un gène,
  chaque colonne à un échantillon. La couleur encode le niveau
  d'expression des gènes.}
\label{fig:micro_array}
\end{figure}

\begin{parts}
\part Pour obtenir une telle représentation, on peut utiliser divers
algorithmes. Comment peut-on décider lequel est le plus approprié~? 
\part Tous les échantillons proviennent du cancer du sein, mais
visiblement, ils ne partagent pas le même profil d'expression. Que
peut-on en déduire sur le cancer du sein en tant que maladie~? Quelles
sont les implications pour le traitement~? 
\end{parts}

\question[2] Apprentissage statistique.
Les machines à vecteurs de support pour des jeux de données
peuvent s'écrire comme un problème d'optimisation : 
\begin{eqnarray*}\label{equ:svm}
\mbox{minimize}_{w,\xi} & & \|w\|^2 + C \sum_{i=1}^{N}\xi_i\\
\mbox{subject to} & & y_i(w^Tx_i + b) \geq 1 - \xi_i \quad i = 1, \ldots, N \\
& & \xi_i \geq 0 \quad i = 1, \ldots, N
\end{eqnarray*}
où $w$ est le vecteur normal de l'hyperplan séparateur, $x_i \in
\mathbb{R}^p$ les vecteurs des descripteurs, $y_i \in \{-1, +1\}$ les
étiquettes. 

\begin{parts}
\part Qu'est-ce que cela veut dire si vous trouvez que pour un descripteur $j$, la valeur
correspondante du vecteur normal $w_j$ a une valeur absolue proche de
0~? 
\part Donner une interprétation des deux termes de la fonction objective ($\|w\|^2$
and $\sum_{i=1}^{N}\xi_i$) et expliquer le rôle du paramètre $C$. 
\part Décrire une façon de fixer le paramètre $C$. 
\end{parts}

\question[2] Innovation th\'erapeutique. La table~\ref{tab:chemoinformatics} pr\'esente les performances en terme de pr\'ecision (``accuracy'', soit la proportion d'examples correctement classifi\'es) de trois algorithmes de classification appliqu\'es \`a trois jeux de donn\'ees QSAR. [Source : V. Svetnik et al. Random forest: a classification and regression tool for compound classification and QSAR modeling. {\it J Chem Info Comp Sci} (2003).]
 
\begin{table}
  \centering
  \begin{tabular}{|l|r|r|r|} \hline
    Data & RF & DT & PLS \\ \hline
    BBB & $0.800$ & $0.732$ & $0.678$ \\
    Estrogen & $0.818$ & $0.753$ & $0.805$ \\
    COX-2 & $0.783$ & $0.742$ & $0.783$ \\ \hline
  \end{tabular}
  \caption{Pr\'ecision (accuracy) de trois m\'ethodes : for\^ets al\'eatoires (Random Forests, RF) ; arbres de d\'ecision (Decision Trees, DT) ; r\'egression des moindres carr\'es partiels (Partial Least Squares, PSL), sur trois jeux de donn\'ees : BBB (Brain-Blood Barrier, capacit\'e \`a traverser la barri\`ere h\'emato-enc\'ephalique) ; Estrogen (capacit\'e \`a se lier \`a un r\'ecepteur \`a estrog\`ene) ; COX-2 (capacit\'e \`a inhiber la cyclooxyg\'enase-2). La pr\'ecision est d\'etermin\'ee par leave-one-out.}
  \label{tab:chemoinformatics}
\end{table}

\begin{parts}
\part D'apr\`es ces donn\'ees, lequel des trois algorithmes est le plus appropri\'e pour conduire des \'etudes des QSAR~?
\part Quel est l'int\'er\^et pour le d\'eveloppement th\'erapeutique de d\'evelopper des m\'ethodes capables de pr\'edire correctement la capacit\'e d'une mol\'ecule \`a traverser la barri\`ere h\'emato-enc\'ephalique~? Sa capacit\'e \`a inhiber une enzyme telle que la cycolooxyg\'enase-2~?
\part Les proportions d'examples positifs vs. n\'egatifs dans les trois jeux de donn\'ees sont $180/145$, $131/101$ et $153/161$. Quel impact cela a-t-il sur la pertinence de l'utilisation de la pr\'ecision comme crit\`ere d'\'evaluation des algorithmes~?
\end{parts}


\section{Questions sur le papier ``Integrative clustering of multiple genomic data types using a joint latent variable model with application to breast and lung cancer subtype analysis.''}

\question[4] Il existe de nombreux algorithmes qui permettent de faire un clustering d'échantillons sur la base de mesures faites sur ces échantillons (par exemple, le niveau d'expression d'un grand nombre de gènes, ou leur nombre de copies).
\begin{parts}
\part Dans quel but cherche-t-on ici à faire du clustering d'échantillons de tumeurs ?
\begin{solution}
  Découvrir des sous-types de cancer.
\end{solution}
\part Quelle est la limitation de ces techniques que l'article se propose de corriger ?
\begin{solution}
  Impossibilité de faire ce clustering sur des sources de données multiples et hétérogènes.
\end{solution}
\part Quelles sont les deux difficultés rencontrées pour mettre en place cette nouvelle méthodologie ?
\begin{solution}
  \begin{itemize}
  \item Modéliser à la fois la covariance entre sources de données et la structure de variance-covariance au sein de chacune des sources.
  \item Réduire la dimension des données simultanément sur différents jeux de données corrélés.
  \end{itemize}
\end{solution}
\part Comment les différentes classes (clusters) d'échantillons sont-ils modélisés dans cette approche ?
\begin{solution}
  Comme des variables latentes / non-observées.
\end{solution}
\end{parts}

\question[2] Donner la dimension et l'interprétation des termes $\bm{W}$, $\bm{\Psi}$, $\bm{X}$ et $\bm{Z^*}$ dans l'Équation 9.
\begin{solution}
  \begin{itemize}
  \item $\bm{W}: (p_1 + p_2 + \dots + p_m) \times (K-1)$: matrice de projection qui associe l'espace des données à celui des $(K-1)$ directions principales sur lesquelles elles sont projetées.
  \item $\bm{\Psi}: (p_1 + p_2 + \dots + p_m) \times (p_1 + p_2 + \dots + p_m)$: matrice bloc-diagonale composée des matrices de covariance $\Psi_1, \dots, \Psi_m$ interne à chaque type de données.
  \item $\bm{X}: (p_1 + p_2 + \dots + p_m) \times n$: données.
  \item $\bm{Z^*}: (K-1) \times n$: version continue de la matrice $\bm{Z}$ qui assigne chaque échantillon à un cluster.
  \end{itemize}
\end{solution}

\question[3] Figure 2.   
\begin{parts}
\part Comment les auteurs utilisent-ils la Figure 2.B pour déterminer le nombre de clusters optimal ?
\part Quelle est la différence entre la Figure 2.A et la Figure 2.D~?
\part Que représentent les 3 lignes sur la Figure 2.E ? 
\end{parts}

\question[4] Figure 3
\begin{parts}
\part TODO
\end{parts}

\end{questions}

%\begin{enumerate}
%\item 
%\end{enumerate}
\end{document}
